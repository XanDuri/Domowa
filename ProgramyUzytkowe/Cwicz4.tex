   \documentclass[a4paper, 12pt]{amsart}   

%\usepackage{amsmath}
\usepackage[T1]{fontenc}
\usepackage{url}
\usepackage{hyperref}
\usepackage{amssymb}
\usepackage[polish]{babel}
\usepackage{enumitem}
\usepackage{listings}







\newtheorem{theorem}{Twierdzenie}

\theoremstyle{definition}
\newtheorem*{definition}{Definicja}


     
\author[A. Muranova]{Anna Muranova}
\title{Ćwiczenie 4}


\begin{document}

\maketitle
\tableofcontents
\section{Wzór na kilka wierszy}
\begin{equation}
|\psi_{1,2}|^{2}=\left\vert 1+\dfrac{LC \lambda^2 }{2}\pm i\sqrt{-LC \lambda^2 -\left( \dfrac{
LC \lambda^2 }{2}\right) ^{2}}\right\vert ^{2}
\end{equation}
\begin{equation}
=\left( 1+\dfrac{LC \lambda^2 }{2}\right)
^{2}-LC \lambda^2 -\left( \dfrac{LC \lambda^2 }{2}\right) ^{2}=1.
\end{equation}

\begin{multline}
|\psi_{1,2}|^{2}=\left\vert 1+\dfrac{LC \lambda^2 }{2}\pm i\sqrt{-LC \lambda^2 -\left( \dfrac{
LC \lambda^2 }{2}\right) ^{2}}\right\vert ^{2}\\
=\left( 1+\dfrac{LC \lambda^2 }{2}\right)
^{2}-LC \lambda^2 -\left( \dfrac{LC \lambda^2 }{2}\right) ^{2}=1.
\end{multline} 

%\begin{dmath}
%|\psi_{1,2}|^{2}=\left\vert 1+\dfrac{LC \lambda^2 }{2}\pm i\sqrt{-LC \lambda^2 -\left( \dfrac{
%LC \lambda^2 }{2}\right) ^{2}}\right\vert ^{2}\\
%=\left( 1+\dfrac{LC \lambda^2 }{2}\right)
%^{2}-LC \lambda^2 -\left( \dfrac{LC \lambda^2 }{2}\right) ^{2}=1.
%\end{dmath}


\begin{align}
|\psi_{1,2}|^{2}=&\left\vert 1+\dfrac{LC \lambda^2 }{2}\pm i\sqrt{-LC \lambda^2 -\left( \dfrac{
LC \lambda^2 }{2}\right) ^{2}}\right\vert ^{2}\label{1}\\
=&\left( 1+\dfrac{LC \lambda^2 }{2}\right)
^{2}-LC \lambda^2 -\left( \dfrac{LC \lambda^2 }{2}\right) ^{2}=1.\label{2}
\end{align}

\begin{equation}
\begin{split}
|\psi_{1,2}|^{2}=&\left\vert 1+\dfrac{LC \lambda^2 }{2}\pm i\sqrt{-LC \lambda^2 -\left( \dfrac{
LC \lambda^2 }{2}\right) ^{2}}\right\vert ^{2}\\
=&\left( 1+\dfrac{LC \lambda^2 }{2}\right)
^{2}-LC \lambda^2 -\left( \dfrac{LC \lambda^2 }{2}\right) ^{2}=1.
\end{split}
\end{equation}
\eqref{1}, \eqref{2}

\section{Układy równań, macierzy oraz tabele}

\begin{equation}
f(x)=
\begin{cases}
\frac{x+5}{12}\mbox{ dla }x>0, \\
x^2+x-5 \mbox{ dla }x\le 0.\\ 
\end{cases}
 \end{equation}

\begin{equation}
f(x)=
\left\{\begin{split}
&\frac{x+5}{12}\mbox{ dla }x>0, \\
&x^2+x-5 \mbox{ dla }x\le 0.\\ 
\end{split}\right.
 \end{equation}


\begin{equation}
\mathbf{A} =
\left( \begin{array}{ccc}
12 & 3 & -10\\
x & 15 & 0 \\
2.5 & -23& 12
\end{array} \right)
\end{equation}


\begin{center}
\begin{tabular}{ c c c }
 cell1 & cell2 & cell3 \\ 
 cell4 & cell5 & cell6 \\  
 cell7 & cell8 & cell9    
\end{tabular}
\end{center}

\begin{center}
\begin{tabular}{|c| c|| l| l||r|} 
 \hline
Numer & Album &\multicolumn{2}{c||}{Imie i Nazwisko}&ocena\\ 
 \hline\hline
 1 & 11111 & Jan & Kowalski &5\\ 
 \hline
 2 &22222  & Grzegorz & Brzęczyszczykiewicz &4.5\\
 \hline
 3 & 12345& Piotr & Wiśniewski&2 \\
 \hline
 4 & 12346& Wojciech & Kowalczyk&2.5 \\
 \hline
 5 & 12347 &  Krystyna & Lewandowska&3 \\
 \hline
\end{tabular}
\end{center}

\section{Pakiet \texttt{listings} dla wpisywania kodów}
\lstset{language=Pascal}
\begin{lstlisting}
function power(x: integer,n: integer): integer;
Var k,a,b:integer;
Begin
	k:=n; a:=1; b:=x;
	while k>0 do begin {Niezmiennik: x^n=a*b^k}
		if k mod 2=0 then begin
			k:=k/2;
			b:=b*b;
		end else begin
			k:=k-1;
			a:=a*b;
		end;
	end;
	power:=a;
End;
\end{lstlisting}

\section{Pakiety \texttt{url} oraz \texttt{hyperref}}
\url{https://ru.overleaf.com/learn/latex/Hyperlinks}\\
\url{https://ru.overleaf.com/learn/latex/Bibliography_management_with_bibtex}\\

\section{Spis literatury}
Niech potrzebujemy książki  \cite{Muranova1}, \cite{Woess}, \cite{Soardi}.

\bibliographystyle{plain}
%\bibliography{myliteratur}

\begin{thebibliography}{10}
\bibitem{Muranova1}
Anna Muranova.
\newblock On the notion of effective impedance.
\newblock {\em arXiv:1905.02047}, \url{http://arxiv.org/abs/arXiv:1905.02047},
  2019.

\bibitem{Soardi}
Paolo~M. Soardi.
\newblock {\em Potential Theory on Infinite Networks}.
\newblock Springer-Verlag, Berlin Heidelberg, 1994.

\bibitem{Woess}
Wolfgang Woess.
\newblock {\em Random Walks on Infinite Graphs and Groups}.
\newblock Number 138 in Cambridge Tracts in Mathematics. Cambridge University
  Press, 2000.


\end{thebibliography}


\end{document}

%%%%%%%%%%%%%%%%%%%%%%%%%%%%%%%%%%%%%%%%%%%%%%%%%%%%%%%%%%%%%%%%%%%%%%%%%%%
