\documentclass[a4paper, 12pt, reqno]{amsart}   

%\usepackage{amsmath}
\usepackage[T1]{fontenc}
\usepackage{amssymb}
\usepackage[polish]{babel}
\usepackage{enumitem}


\newcommand{\A}{\mathfrak {A}}
%\renewcommand{\P}{\mathcal {P}}
\renewcommand{\Re}{\operatorname{Re}}%real part
\renewcommand{\Im}{\operatorname{Im}}%imaginary part
 \newcommand{\tg}{\operatorname{tg}}%tangens
\newcommand{\pd}[2]{\frac{\partial #1}{\partial #2}}%partial_derivative
\newcommand{\pc}[2]{\partial_{#1}#2}%partial_derivative
\newcommand{\abs}[1]{|#1|}
\newcommand{\norm}[1]{\|#1\|}

\newtheorem{theorem}{Twierdzenie}

\theoremstyle{definition}
\newtheorem*{definition}{Definicja}


     
\author[A. Muranova]{Anna Muranova}
\title{Ró\.zny rzeczy matematyczny}



\begin{document}


\maketitle

$\parallel, \not\parallel, \nparallel$
\section{Matematyczne kroje pisma}
$\mathrm{ABCdef}$\\
$\mathit{ABCdef}$\\
$ \mathnormal{ABCdef}$\\
$\mathfrak{ABCdef}$\\  
$\mathcal{ABC}$\\
$\mathbb{ABC}$\\


The main Theorem says that $\mathcal{P}\left( \lambda \right) :=\lim\limits_{n\rightarrow\infty} \mathcal{P}_{n}\left(
\lambda \right) $ exists and is a holomorphic function of $\lambda $ in the
domain $\left\{ \mathrm{Re}\;\lambda >0\right\} $ as well as in some other
regions. 


\section{Operatory}
$\A, \P$\\
$ \Re, \Im$\\
$\tg$\\
$\pd{x}{y}$\\
$\abs{y}$\\
$\norm x$\\
$\pc {x}{f}$


\section{Srodowiska theorem, proof}

\begin{theorem}[Wielkie twierdzenie Fermata]\label{F}
dla liczby naturalnej $n>2$ nie istnieją takie liczby naturalne dodatnie $x,y,z$ które spełniałyby równanie 
\begin{equation}\label{1}
{\displaystyle x^{n}+y^{n}=z^{n}.} 
\end{equation}
\end{theorem}
\begin{proof}
W rzeczywistości dowód twierdzenia Fermata przeprowadzony przez Wilesa ma dosyć długą historię.
\end{proof}


  

Wielu matematyków nadal szuka dowodu Twierdzenia \ref{F} na bazie teorii liczb.

\begin{definition}
\textit{Trójka pitagorejska} -- trzy liczby naturalne $x,y,z$ spełniające tzw. równanie Pitagorasa \eqref{1}.
\end{definition}

\section{Listy i spisy}
Lista zakupów
\begin{itemize}[label=$\heartsuit$]
  \item Czekolada.
  \item Kawa.
  \item Mleko.
\end{itemize}

Lista zakupów
\begin{enumerate}
  \item Czekolada.
  \item Kawa.
  \item Mleko.
\end{enumerate}

Lista zakupów
\begin{enumerate}
  \item słodyczy:
		\begin{enumerate}
		  \item cukierki,
		  \item dżem,
		  \item nutella;
		\end{enumerate}
  \item napoje:
		\begin{enumerate}
		  \item kawa,
		  \item herbata,
		  \item coca-cola;
		\end{enumerate}
  \item warzywa:
		\begin{enumerate}
		  \item dynia,
		  \item ogórki,
		\end{enumerate}
 \item owoce:
		\begin{enumerate}
		  \item banana,
		  \item jabłka,
  		\item jagody
		          \begin{enumerate}
				  \item borówki,
				  \item maliny.
				\end{enumerate}
		\end{enumerate}
\item pizza
\end{enumerate}


\end{document}

%%%%%%%%%%%%%%%%%%%%%%%%%%%%%%%%%%%%%%%%%%%%%%%%%%%%%%%%%%%%%%%%%%%%%%%%%%%
